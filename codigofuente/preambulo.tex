
\usepackage[T1]{fontenc} %acentos en español
\usepackage[utf8]{inputenc} 
\usepackage[spanish]{babel} %para colocar documento en español
\usepackage{bm} %para colocar negritas
\usepackage{float} %situar figuras en las partes que quiera
\usepackage{subfigure} %para colocar varias figuras
\usepackage{titling} %para dar formato a titulo
\usepackage{titlesec}
\usepackage{natbib} %para bibliografías
\usepackage{parskip} %para espacio en blanco despues de un parrafo
\usepackage{graphicx} %para gráficos y figuras
\graphicspath{imagenes/} %direccion carpeta imagenes
\usepackage{fancyhdr} %para encabezado y pie de pagina
\usepackage{vmargin}
\setmarginsrb{3 cm}{2.5 cm}{3 cm}{2.5 cm}{1 cm}{1.5 cm}{1 cm}{1.5 cm}

\usepackage[natbibapa]{apacite}
%\usepackage{apacite}

%parteexperimental
\usepackage{natbib}
\usepackage{url}
%%%%%%%%%%%%%%%%%%%%%%%%%%%%

\title{Ionic Framework}
\author{Boris Nilo Moreno}
\date{9 de mayo de 2019}

\makeatletter
\let\eltitulo\@title
\let\elautor\@author
\let\lafecha\@date
\makeatother

\pagestyle{fancy}
\fancyhf{}
\rhead{\includegraphics[scale =0.4 ]{imagenes/Logo_DuocUC.png}}
\lhead{\eltitulo}
\cfoot{\thepage}

%Para agregar referencias al indice 
\let\OLDthebibliography=\thebibliography
\def\thebibliography#1{\OLDthebibliography{#1}%
\addcontentsline{toc}{section}{\refname}}

