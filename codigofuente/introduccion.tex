\section{Introducción}
\parindent=5mm

En el mundo actual, los artefactos electrónicos han presentado un gran avance, ya que los hay de diferentes tamaños, formas, y algunos, con capacidad de operar como dispositivos inteligentes. Disponemos de Smartphones, Smartwatches, Televisores, que funcionan en base a sistemas operativos (S.O.), a los cuales se les pueden instalar aplicaciones (APP). Estas, se relacionan con el S.O, a través de un lenguaje nativo de programación, que les permiten interactuar y entenderse con los componentes de software y hardware internos del dispositivo. En el caso de los SmartPhones, los S.O. que están masificados son dos: IOS y ANDROID, y sus aplicaciones están escritas en los lenguajes nativos Objetive-C y Java, respectivamente.

Desarrollar una APP para un Sistema, requiere de un grado de especialización en el lenguaje de programación nativo perteneciente al S.O. Sin embargo, eso eleva mucho los costos de desarrollo. Es ahí donde entra en juego una aplicación híbrida. Según \citet{angulo}, una aplicación híbrida se basa en el desarrollo de una página móvil con capacidad para manejar los elementos nativos del dispositivo (cámara y GPS, entre otros). Es decir que, una persona con conocimiento en diseño web, esta habilitada para desarrollar una APP, que satisfaga los requerimientos de un grupo de usuarios o una empresa. Por tanto, una aplicación híbrida, disminuye los costos de especialización, los costos de mantenimiento y el tiempo de desarrollo de aplicaciones. 

Dicho lo anterior, la plataforma para la realización de aplicaciones móviles híbridas más popular es Ionic Framework, la cual es de código libre, y permite el diseño de interfaces para todos los diseñadores web.

En el presente informe se expone una breve definición de tecnicismos para una mejor comprensión. A partir de estos, se responde qué es Ionic, cómo funciona y sus ventajas y desventajas. Para terminar, en la conclusión se hará un resumen de esta tecnología que permite que cualquier persona con un conocimiento de diseño y programación, pueda crear sus propias aplicaciones.
