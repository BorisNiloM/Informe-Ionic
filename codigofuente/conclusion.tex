\section{Conclusión}

Una de las formas más rápidas y fáciles de desarrollar una aplicación para teléfonos móviles es Ionic Framework. Esta plantea una solución directa al problema del desarrollo multiplataforma, ya que, una vez creada la aplicación, funciona en todos los dispositivos. Además, como utiliza tecnología web, se puede tener conocimientos básicos de HTML, CSS y Javascript, los cuales, hoy en día, se obtienen siguiendo algún curso en youtube o leyendo libros afínes. Pero también, sin tener conocimientos de nada, Ionic Framework es de fácil aprendizaje por sí solo. Lo más importante es seguir los lineamientos y las directivas de trabajo, ya que así se pueden lograr resultados aceptables.

El presente documento informativo, abarco el tema de Ionic, desde una definición de conceptos básicos, para poder así explicar cómo funciona y sus ventajas y desventajas. La finalidad de esto, es dar a conocer que el desarrollo de aplicaciones, no solo incumbe a personas con conocimientos avanzados en programación, ya que cualquier persona es capaz de realizar una app que pueda dar solución a un problema.